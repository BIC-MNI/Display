\documentstyle[david,psfig]{article}

\title{{\bf Display} - Program for Display and Segmentation of Surfaces and Volumes}
\author{David MacDonald}

\newcommand{\ident}[1]{{\bf\tt #1}\ }
\newcommand{\display}{\ident{display}}
\newcommand{\menu}[1]{{\scriptsize \fbox{\bf #1}}}
\newcommand{\menutwo}[2]{{\scriptsize \fbox{\bf #1}/\fbox{\bf #2}}}
\newcommand{\menuthree}[3]{{\scriptsize \fbox{\bf #1}/\fbox{\bf #2}/\fbox{\bf #3}}}

\psdraft

\begin{document}

\maketitle

\newpage

\tableofcontents

\newpage

\section{Introduction}

\display is a program originally designed to display and manipulate
three dimensional objects,
mainly human cortical surfaces and sulcal curves.  It has since evolved to
include MRI and PET volume display and segmentation and a variety of other
features.  The user interface is a non-standard menu oriented system, based
on keystrokes and mouse selection.

\section{Running \display}

\display runs only on Silicon Graphics (SGI) workstations.
The command syntax is:

\begin{verbatim}
    display  [file1]  [file2]  ... [filen]
\end{verbatim}

where each file is one of:

\begin{itemize}

\item[Volume:]  contains an MRI, PET, or other 3D volume, and ends
                    in \ident{.mnc},

\item[3D object:]  contains 3D surfaces, lines, or other objects, and
                    ends in \ident{.obj},

\item[Tags:]  contains a list of 3D points, such as those chosen from a volume.
              and ends in \ident{.tag}.
\end{itemize}

\subsection{MRI Files at the MNI}

The following are the standard available datasets:

\begin{itemize}
\item[Normal MRI] 305 normal brains in the files:  \ident{/avgbrain/brain/autoreg/filename.mnc}, and
\item[Average MRI] the average of the 305 brains in the file
\ident{/avgbrain/brain/images/norm\_avg\_305\_mri256.mnc}, or a lower
resolution version in \ident{/avgbrain/brain/images/norm\_avg\_305\_mri.mnc}.
\end{itemize}

Note that the above files are generally compressed.  If the file ends in
\ .z, then it is compressed and the full name of the file, including the
\ .z, must be specified to display.

\section{\display Windows}

\psfigure{displaywindows.ps}{7in}{top left: 3D window, top right: slice window,
centre: menu window, bottom: text entry window.}{windows}

Figure \ref{windows} shows the four windows used by \display (three of which
are created when the program is run):

\begin{itemize}
\item[3D window:]  contains the 3D objects, such as surfaces, lines,
                  and markers,
\item[slice window:]  shows up to 4 views of slices through the volume
                      (if a volume is loaded),
\item[menu window:]  shows the current menu configuration,
\item[text entry:]  this is the shell window or xterm from which \display was
                   run and in which all text must be typed, e.g. filenames,
                   numerical parameters, etc.
\end{itemize}

\section{Menu and Interaction System}

The menu window (centre in Figure \ref{windows}) represents the layout
of the left side of the keyboard.  A menu entry is selected by hitting the
corresponding key in any of the 3 display windows (not the text window)
or pointing to the entry
in the menu window with the mouse and clicking the left button.
The space bar or middle mouse
button will pop back one level in the menu.  If the display of a menu
entry is ``grayed out'' (in a less visible colour), then this signals that the
entry is not valid in the current context and cannot be selected.  In the
remainder of this document
the following convention will be used to refer to menu selection:
\menutwo{View}{Reset} means select the \ident{View} menu, then
select the \ident{Reset} button
from within the view menu.  Whenever text must be typed, such as in prompts
for filenames, size and width parameters, etc., the mouse must be moved to the
text entry window and the values typed, followed by the return key.
Some menu items cannot be selected by the mouse because the mouse must be
used to point to the object of the action, usually one of the 4 volume slices
in the slice window.
In these cases, the mouse must be positioned on the relevant slice in
the slice window and the
keyboard character corresponding to the menu entry pressed.
When the user is prompted to type in a colour, either the name of a colour,
such as \ident{red}, \ident{yellow}, or \ident{pink}, or a numerical
red-green-blue colour,
such as \mbox{\ident{0.3 0.7 0.7}} may be typed in.

\subsection{Menu Window}

The menu window contains the representation of the partial keyboard, and
displays the object hierarchy.  All objects loaded and created in \display,
with the exception of volumes are displayed in the object hierarchy.  The
arrow keys are used to move around in this hierarchy and to select the
current object.  Or clicking on an object with the left mouse button
will make it the current selection.  On the lower left is displayed the
name of the currently selected menu.

\subsection{3D Window}

The 3D window shows three dimensional objects such as surfaces and
lines, with lighting and camera control.  In the 3D window, the following
mouse operations are available:

\begin{itemize}
\item[Left Button]  Selects a position on a surface or a marker (tag point).
\item[Middle Button]  Rotates, translates, or magnifies the 3D view, depending
                      on the current mode selected from the view 3D menu.
\end{itemize}

\subsection{Slice Window}

The slice window displays slices of the volume, with various colour coding
options.  The slice window carries around another volume of the same size
as the loaded volume, which is referred to as the {\bf label volume}.
This functions as a coloured overlay for storing and displaying the results
of manual segmentation of the volume.
In the slice window, the following mouse operations are available:

\begin{itemize}
\item[Left Button]  Sets the position of the 4 slices.
\item[Middle Button]  Moving the mouse up or down while holding down the
                      middle button changes the current volume slice.
\item[Right Button]  Paints a region of the volume slice with the current
                     brush size.
\item[Alt-Left Button]  While holding down the \ident{shift}, \ident{ctrl},
                        or \ident{alt} key, the left button translates the
                        position of the slice within the viewport.
\item[Alt-Right Button]  While holding down the \ident{shift}, \ident{ctrl},
                        or \ident{alt} key, holding down the middle button
                        and moving the mouse up or down changes the
                        magnification of the current slice.
\item[Colour Bar]  The limits of the colour bar can be moved by pressing
                   the left mouse button on the bar near the desired limit.
                   Pressing the middle button on the colour bar will allow
                   both limits to be moved simultaneously.
\end{itemize}

\section{Complete Menu Reference}

\psfigure{menu.ps}{7in}{Top level menu}{menu}

The following is a complete listing of every menu selection, with a
short description of each.
Since this program is also partly a research tool, a few selections are
not relevant to most users, and will simply be described as not for general
use.  Fig. \ref{menu} shows the top level menu that is presented upon
program start up, or after popping the menu to the top level.

\subsection{Current Object Menu}

The 3D graphics objects, visible in the 3D window, are also presented
in a hierarchical form in the menu window, as a tree structure with
each element of the tree being the name of a 3D graphics object.
There are 4 keys that can be used to navigate through the list, where
the currently selected object is always displayed with a surrounding
box:

\begin{description}
\item[\menu{Up Arrow}]  Sets the current object to the one before
                the current object.
\item[\menu{Down Arrow}]  Sets the current object to the one after
                the current object.
\item[\menu{Left Arrow}]  Sets the current object to the one above
                the current object.
\item[\menu{Right Arrow}]  Sets the current object to the one below
                the current object, if the current object is a model.
\end{description}

\subsection{File Menu}

\begin{description}
\item[\menutwo{File}{Load File}]  Prompts for a filename and loads the file.
        If it is a volume file (ends in \ident{.mnc}), then the
        slice window is opened, if not already opened.
\item[\menutwo{File}{Save File}]  Prompts for a filename and saves to the file
        the current object, or all objects under the current object, if it is
        a model object.  This selection cannot be used to save a volume
        file (\ident{.mnc})
\item[\menutwo{File}{Load Labels .mnc}]  Prompts for a filename and
        loads the volume file as the label volume for the slice window
        volume.
        The volume loaded must be exactly the same size as the slice
        window volume.
\item[\menutwo{File}{Save Labels .mnc}]  Prompts for a filename and
        saves the label volume.
\item[\menutwo{File}{Load Labels .tag}]  Prompts for a filename and
        loads the tags file into the label volume.  Note that it does not
        clear the label volume, so the resulting labels are the union
        of the current labels and the loaded tags.  Also, it does not check
        that this is a valid label file, e.g., whether sizes match.
\item[\menutwo{File}{Save Labels .tag}]  Prompts for a filename and
        saves the label volume as a tag file.  For large regions
        this may create very large ascii files, and it may be more
        efficient to save as \ident{.mnc}.
\item[\menutwo{File}{Save Markers as .tag}]  Prompts for a filename and
        saves any markers at or under the current object to the file
        in tag file format.

\item[\menutwo{File}{Load Poly Visib.}]  Not for general use.
\item[\menutwo{File}{Save Poly Visib.}]  Not for general use.
\item[\menutwo{File}{Save Bintree}]  Not for general use.
\item[\menutwo{File}{Load Bintree}]  Not for general use.
\end{description}

\subsection{Slice Menu}

\begin{description}
\item[\menu{+ Slice}]  Changes the current slice to the next slice.
\item[\menu{- Slice}]  Changes the current slice to the previous slice.
\item[\menutwo{Slice}{Reset Slice View}]  If the mouse is in the slice
   window and is pointing to one of the 4 slices, then the view for
   that slice is reset and resized to fit the viewport.
\item[\menutwo{Slice}{Toggle Slice Visib}]  If the mouse is in the slice
   window and is pointing to one of the 4 slices, then the visibility
   of that slice is toggled.
\item[\menutwo{Slice}{Box Filter Volume}]  Prompts for 3 box filter
       widths and another specifier.  If the specifier is the
       character ``w'', then the
       filter widths are assumed to be in world coordinates, otherwise
       it is assumed to be voxel coordinates.  The volume is resampled
       using a box filter.  Giving the values ``0 0 0 v'' resets to the
       original loaded volume.
\item[\menutwo{Slice}{Resample Volume}]  Prompts for 3 size parameter and
       resamples the volume to the given size.  Entering 0 0 0 resets to
       the original volume.  Creates a smaller volume by box
       filtering.  It is preferable to use \menutwo{Slice}{Box Filter Volume}.
\item[\menutwo{Slice}{Create 3D Slice}]  Creates a rendering of the slice
       pointed to by the mouse, in the 3D window.
\item[\menutwo{Slice}{Recompute Histogram}]  Creates a histogram of
       values in the volume, which it displays in the slice window near
       the colour coding bar.
       If the mouse is pointing to a slice in the slice window, then a
       histogram of that slice only is displayed.
\item[\menutwo{Slice}{Histogram of Label}]  Sames as
       \menutwo{Slice}{Recompute Histogram}, but only includes values
       which are in labeled (segmented) regions.
\item[\menutwo{Slice}{Slice DblBuf:}]  Toggles between single and double buffer
       mode for the slice window.  Single buffer mode provides maximum
       colour resolution, but flashes when updating.  Double buffer
       mode has less colour resolution but updates smoothly.  Single
       buffer mode is generally only used when taking snapshots of the
       window, where maximum colour resolution is desirable.
\item[\menutwo{Slice}{Toggle Plane Visib}]  The fourth slice view in
       the slice window is an arbitrarily oriented plane.  This menu
       selection toggles the visibility of this cross section plane in
       the 3D window.
\item[\menutwo{Slice}{Set Current Arb. View}]  Each of the four slices can
       be oriented to an arbitrary angle.  By pointing to one of the slices
       and selecting this menu option, the user can set the current view,
       which is the one that the following slicing commands operate on.
\item[\menutwo{Slice}{Toggle Slice Crs-Sect}]  Turns on and off the cross
       section visibility of the current view slice.
\item[\menutwo{Slice}{Rotate Slice}]  After selecting this entry,
       the middle mouse button in the 3D window will control rotation
       of the arbitrarily oriented slice, updating the current
       view slice in the slice window.
\item[\menutwo{Slice}{Pick Slice Angle}]  Instead of rotating the
       slice in the 3D window, the orientation of the current view slice
       can be chosen from within the slice window.  After
       selecting this menu entry, the left
       mouse button selects a position on one of the other slices.  The
       line through the slice cursur and this position is used to define
       the slice plane of the current view slice.
\item[\menutwo{Slice}{Toggle Slice Anchor}]  Turns on and off the anchoring
       of the current view slice to pass through the cross section of the
       current slice view.  This is used to constrain a slice plane to pass
       through a given vector.
\item[\menutwo{Slice}{Print Origin}]  Prints the cursor location in world
       space to the text window.
\item[\menutwo{Slice}{Print Plane Normal}]  Prints the world space normal of
       the slice currently under the mouse.
\item[\menutwo{Slice}{Type In Origin}]  Prompts the user to type in a world
       space x, y, and z, and moves the cursor to this point.
\item[\menutwo{Slice}{Type In Plane Normal}]  If the mouse is pointing to
       one of the four slices, prompts the user to type in a world
       space x, y, and z, and orients the slice plane to this normal.
\end{description}

\subsection{Colour Coding Menu}

\begin{description}
\item[\menutwo{Colour Coding}{Spectral}]  Selects the spectral colour coding
                                          method for the slice volume.
\item[\menutwo{Colour Coding}{Gray Scale}]  Selects the gray scale colour coding
                                          method for the slice volume.
\item[\menutwo{Colour Coding}{Hot Metal}]  Selects the hot metal colour coding
                                          method for the slice volume.
\item[\menutwo{Colour Coding}{Contour}]  Selects the rarely used contour
                                    colour coding method for the slice volume.
\item[\menutwo{Colour Coding}{Range}]  Prompts the user to type in the lower
                                    and upper colour coding limits.
\item[\menutwo{Colour Coding}{Under Colour}]  Prompts the user to type in
                                   the colour for values below the low limit.
\item[\menutwo{Colour Coding}{Over Colour}]  Prompts the user to type in
                                   the colour for values above the high limit.
\item[\menutwo{Colour Coding}{Nearest Neighbour}]  Sets the filter type of
                       the slice under the mouse to nearest neighbour (this is
                       the default).
\item[\menutwo{Colour Coding}{Linear Int Filter}]  Sets the filter type of
                       the slice under the mouse to linear interpolation
                       between the two nearest slices.
\item[\menutwo{Colour Coding}{Box Filter}]  Sets the filter type of
                       the slice under the mouse to a box filter.
\item[\menutwo{Colour Coding}{Triangle Filter}]  Sets the filter type of
                       the slice under the mouse to a triangle filter.
\item[\menutwo{Colour Coding}{Gaussian Filter}]  Sets the filter type of
                       the slice under the mouse to a gaussian filter.
\item[\menutwo{Colour Coding}{Filter Width}]  Prompts the user for
                      the filter full width half max.  Only applies to box,
                      triangle, and gaussian filters.
\item[\menutwo{Colour Coding}{Label Ratio}]  Prompts the user to type in the
                      intensity of the coloured labels superimposed on the 
                      volume slices ($0 <= value <= 1$).
\item[\menutwo{Colour Coding}{Show Labels}]  Toggles between showing
                      the label volume superimposed on the loaded
                      volume or not showing it.
\item[\menutwo{Colour Coding}{Set Paint Lbl Colr}]  Prompts for a
                      label value and a colour, and sets the displayed
                      colour of this label.
\item[\menutwo{Colour Coding}{Num Labels}]  Prompts for the number
                      of labels desired, and recreates the label volume
                      with this number.  Depending on the number of labels,
                      the volume may be a byte, short, or long valued volume.
                      Note that this effectively clears the label volume.
\item[\menutwo{Colour Coding}{Colour Code Object}]  Changes the
     colours of the current object according to the current colour
     coding parameters of the volume.
\end{description}

\subsection{Segmenting Menu}

This menu contains all the controls for modifying the label volume,
which is overlaid on the loaded volume.  At each voxel in the loaded
volume, there is an associated value in the range 0 to 255, which is
stored in the label volume.  Segmenting consists of painting regions
of the label volume to change the values from the default of 0.
When the right mouse button is pressed over a slice in the slice
window, the current paint label value is painted into the label
volume.  If the \ident{control} key or \ident{shift} key is held down
at the same time, erasing will be performed, by storing the value 0.

\begin{description}
\item[\menutwo{Segmenting}{Clear All Labels}]  Sets all labels to 0.
\item[\menutwo{Segmenting}{Set Paint Label}]  Sets the current label
    used for painting, must be between 0 and 255, where 0 erases.
\item[\menutwo{Segmenting}{XY Radius}]  Prompts for the in-slice 
    brush radius used in painting on the volume slices (right mouse button).
\item[\menutwo{Segmenting}{Out-Plane Radius}]  Prompts for the brush
    radius in the direction perpendicular to the slice plane.  This
    defaults to zero, for slice-by-slice painting.
\item[\menutwo{Segmenting}{Label Voxel}]  Sets the label of the voxel
    underneath the mouse to the current paint label.
\item[\menutwo{Segmenting}{Clear Voxel}]  Sets the label of the voxel
    underneath the mouse to 0.
\item[\menutwo{Segmenting}{Label Slice}]  Sets the label of the 
    entire slice pointed to by the mouse to the current paint label.
\item[\menutwo{Segmenting}{Clear Slice}]  Sets the label of the 
    entire slice pointed to by the mouse to 0.
\item[\menutwo{Segmenting}{Set Threshold}]  Prompts for a minimum and
    maximum volume value, and uses these limits for subsequent segmentation
    operations.  When painting with the right mouse button, only
    voxels whose values are within this range are affected.  By
    default, there is no segmenting threshold, which can be specified
    by a max value which is less than the min value.
\item[\menutwo{Segmenting}{Label Fill}]  If the mouse is pointing
    to a voxel which is within the selected segmenting range and which
    has a label not equal to the current paint label, then all similar
    voxels on this slice connected
    to the starting voxel are assigned the current paint label, by a
    flood fill algorithm.
\item[\menutwo{Segmenting}{Label Fill No Thrs}]  Sames as
    \menutwo{Segmenting}{Label Fill}, except ignoring the threshold.
    If the mouse is pointing to a voxel has a label not equal to the
    current paint label, then all
    zero-labeled voxels on this slice connected
    to the starting voxel are assigned the current paint label, by a
    flood fill algorithm.
\item[\menutwo{Segmenting}{Clear Fill}]  If the mouse is pointing
    to a voxel which is within the selected segmenting range and which
    has a nonzero label, then all similar voxels on this slice connected
    to the starting voxel are assigned the paint label 0, by a
    flood fill algorithm.
\item[\menutwo{Segmenting}{Copy from Rt/Sup/Ant}]  If the mouse is
    pointing to a slice in the slice window, then the labels of the
    neighbouring slice are copied to this slice.
    For transverse
    slices, the neighbour is the slice just superior to this one.
    For coronal
    slices, the neighbour is the slice just anterior to this one.
    For sagittal
    slices, the neighbour is the slice just to the right of this one.
\item[\menutwo{Segmenting}{Copy from Lt/Inf/Pos}]  If the mouse is
    pointing to a slice in the slice window, then the labels of the
    neighbouring slice are copied to this slice.
    For transverse
    slices, the neighbour is the slice just inferior to this one.
    For coronal
    slices, the neighbour is the slice just posterior to this one.
    For sagittal
    slices, the neighbour is the slice just to the left of this one.
\item[\menutwo{Segmenting}{Fill 3D}]  If the mouse is pointing to a
    voxel which has a label which is not equal to the current paint
    label and is within the threshold, then
    all similar voxels in the entire volume which are connected to
    this one are assigned the current paint label.
    This may take a few seconds to a minute.
\item[\menutwo{Segmenting}{Expand 3D}]  This is a grow operation,
    where all zero-labeled voxels in the volume which are next to
    a voxel of the current paint label are assigned the label.
    This takes a few seconds to perform.
\item[\menutwo{Segmenting}{Calculate Volume}]  Calculates the
    total volume of all voxels which have the current paint label.
    This takes a few seconds to perform.
\item[\menutwo{Segmenting}{Change Labels}]  Prompts for a source
    and destination label value, as well as a volume value minimum and
    maximum value.  All voxels which have the source label value and
    are within the specified value range are changed to have the
    destination label.  If the maximum value specified is less than
    the minimum value, then this range is ignored, and the operation
    simply changes all occurrences of the source label to the
    destination label.
\item[\menutwo{Segmenting}{Trans \^\ }]  Moves all labels in the
    upwards direction of the slice pointed to by the mouse.  May take
    up to a minute to perform.
\item[\menutwo{Segmenting}{Trans v}]  Moves all labels in the
    down direction of the slice pointed to by the mouse.  May take
    up to a minute to perform.
\item[\menutwo{Segmenting}{Trans $<$}]  Moves all labels in the
    left direction of the slice pointed to by the mouse.  May take
    up to a minute to perform.
\item[\menutwo{Segmenting}{Trans $>$}]  Moves all labels in the
    right direction of the slice pointed to by the mouse.  May take
    up to a minute to perform.
\item[\menutwo{Segmenting}{Big Translate}]  Prompts for 3 voxel
    offsets and moves all labels by this amount.
    May take up to a minute to perform.
\end{description}

\subsection{Create Surface}

This menu is used to create surfaces from the volume.  Surfaces created
from this menu should subsequently have their appearance smoothed by
the use of \menutwo{Polygons}{Compute Normals} or
\menutwo{Polygons}{Average Normals}.

\begin{description}
\item[\menutwo{Create Surface}{Volume Isosurface}]  Prompts for a
    value, then starts extracting a polygonal isosurface from near the slice
    cursor.  The isosurface is extracted to the 3D window and updated
    periodically, while all program operations are still functional.
\item[\menutwo{Create Surface}{Extracting}]  If an isosurface extraction
    is in progress, this button toggles the extraction process on and
    off.  This allows suspension and resumption of isosurface extraction.
\item[\menutwo{Create Surface}{Reset Surface}]  Clears the current
    isosurface extraction, if any.  This must be selected before starting
    another isosurface creation method, e.g.,
    \menutwo{Create Surface}{Volume Isosurface}.
\item[\menutwo{Create Surface}{Make Perm}]  If an isosurface extraction
    is in progress, the currently extracted isosurface is made a permanent
    member of the 3D object hierarchy.
\item[\menutwo{Create Surface}{Volume Bin-Isosurf}]  Prompts for two
    values, specifying a volume value range, then starts extracting a
    polygonal isosurface from near the slice
    cursor.  This differs from \menutwo{Create Surface}{Volume Isosurface},
    in that voxels are classified as in or out and the isosurface points are
    exactly half way between an inside voxel and outside voxel.  Typically,
    this is only useful for segmented volumes, otherwise the user
    should use \menutwo{Create Surface}{Volume Isosurface}, which
    results in a smoother surface.
    The surface is extracted to the 3D window and updated
    periodically, while all program operations are still functional.
\item[\menutwo{Create Surface}{Volume Voxelate}]  Prompts for two
    values, specifying a volume value range, then creates a voxelated
    surface.  A voxelated surface is one composed entirely of rectangular
    faces of voxels, the boundaries between inside voxels and outside
    voxels, as defined by the value range specified.  This may take
    up to a minute to perform.
\item[\menutwo{Create Surface}{Label Bin-Isosurf}]  Sames as
    \menutwo{Create Surface}{Volume Bin-Isosurf} except operates on
    the label volume.  For instance, if the user has painted a region
    with label 1, then to create a 3D isosurface of this region, select
    this menu option, and type in ``1 1'' for the min and max values.
\item[\menutwo{Create Surface}{Label Voxelate}]  Sames as
    \menutwo{Create Surface}{Volume Voxelate} except operates on
    the label volume.  For instance, if the user has painted a region
    with label 1, then to create a 3D voxelated surface of this region,
    select this menu option, and type in ``1 1'' for the min and max values.
\item[\menutwo{Create Surface}{X Dist}]  Prompts the user for the maximum
    distance in the X direction from the starting point of the isosurface
    that the isosurface is allowed to expand.  By default, this is disabled
    by setting it to a very large value.
\item[\menutwo{Create Surface}{Y Dist}]  Prompts the user for the maximum
    distance in the Y direction from the starting point of the isosurface
    that the isosurface is allowed to expand.  By default, this is disabled
    by setting it to a very large value.
\item[\menutwo{Create Surface}{Z Dist}]  Prompts the user for the maximum
    distance in the Z direction from the starting point of the isosurface
    that the isosurface is allowed to expand.  By default, this is disabled
    by setting it to a very large value.
\end{description}

\subsection{Atlas Menu}

This menu controls the display of the Talairach atlas overlaid on the
the slice views.  The colour atlas book has been scanned into a digital
format and can be overlaid on any volume for reference and comparison
purposes.

\begin{description}
\item[\menutwo{Atlas}{Atlas State}]  Toggles between displaying scanned images
        of the Talairach Atlas Book superimposed on the volume slices.  The
        first time this is pressed there will be a delay of about 2 minutes
        while the data is read in.
\item[\menutwo{Atlas}{Set Opacity}]  Prompts for the opacity of the atlas.
        A value of 1 will not show the volume slice through the atlas, while
        values closer to 0 will show a more transparent atlas on top of the
        volume.
\item[\menutwo{Atlas}{Set Tolerance X}]  Sets the distance from the current
        sagittal slice that atlas pages must be within in order to be
        displayed.
\item[\menutwo{Atlas}{Set Tolerance Y}]  Sets the distance from the current
        coronal slice that atlas pages must be within in order to be
        displayed.
\item[\menutwo{Atlas}{Set Tolerance Z}]  Sets the distance from the current
        transverse slice that atlas pages must be within in order to be
        displayed.
\item[\menutwo{Atlas}{Set Transparent Threshold}]  Not for general use.
\item[\menutwo{Atlas}{Flip X}]  Flips the sagittal atlas pages.
\item[\menutwo{Atlas}{Flip Y}]  Flips the coronal atlas pages.
\item[\menutwo{Atlas}{Flip Z}]  Flips the transverse atlas pages.
\end{description}

\subsection{3D View Menu}

\begin{description}
\item[\menutwo{3D View}{Magnify 3D}]
      Sets the mode where the middle mouse button and left/right movement
      in the 3D window magnify the view.
      Once the 3D view menu is chosen, this button will usually be
      available, even after popping out of the 3D view menu.
\item[\menutwo{3D View}{Translate 3D}]  
      Sets the mode where the middle mouse button and movement
      in the 3D window translate the objects in the view.
      Once the 3D view menu is chosen, this button will usually be
      available, even after popping out of the 3D view menu.
\item[\menutwo{3D View}{Rotate 3D}]  Sets the mode where
      the middle mouse button and movement in the 3D window rotate
      the objects in a virtual trackball fashion.
      Once the 3D view menu is chosen, this button will usually be
      available, even after popping out of the 3D view menu.
\item[\menutwo{3D View}{Front View}]  Selects a view of the front of
      the objects.
\item[\menutwo{3D View}{Back View}]  Selects a view of the back of
      the objects.
\item[\menutwo{3D View}{Left View}]  Selects a view of the left side of
      the objects.
\item[\menutwo{3D View}{Right View}]  Selects a view of the right side of
      the objects.
\item[\menutwo{3D View}{Top View}]  Selects a view of the top side of
      the objects.
\item[\menutwo{3D View}{Bottom View}]  Selects a view of the bottom side of
      the objects.
\item[\menutwo{3D View}{Left Tilted View}]  Selects a view of the left side of
      the objects, tilted forward.
\item[\menutwo{3D View}{Right Tilted View}]  Selects a view of the right side of
      the objects, tilted forward.
\item[\menutwo{3D View}{Reset View}]  Resets the view to a top view.
\item[\menutwo{3D View}{Fit View}]  Without changing the view direction,
      magnifies the objects to just fit inside the window.  Useful when
      the user desires to see the full extent of all objects.
\item[\menutwo{3D View}{Parallel/Perspective}]  Toggles between a
      parallel and perspective view of the 3D objects.
\item[\menutwo{3D View}{Save Image}]  Prompts for a filename and saves the
      contents of the 3D window to the file, in .rgb format.  Before doing
      this, make sure the window is up to date and not in the process of
      drawing.
\item[\menutwo{3D View}{Film Loop}]  Creates a movie of the 3D window.
      Prompts for a filename prefix, an axis index (0-2), and a number
      of frames.  The objects are spun around the specified index saving
      a separate frame to file for each increment.
\item[\menutwo{3D View}{Front Plane}]  Not for general use.
\item[\menutwo{3D View}{Back Plane}]  Not for general use.
\item[\menutwo{3D View}{Toggle Stereo}]  Not for general use.
\item[\menutwo{3D View}{Eye Width}]  Not for general use.
\item[\menutwo{3D View}{Pick View}]  Not for general use.
\end{description}

\subsection{Objects Menu}

\begin{description}
\item[\menuthree{Objects}{Delete Object}{Really Delete}]  Deletes the current
        object.
\item[\menuthree{Objects}{Delete Object}{Don't Delete}]  Changes mind and
        does not delete the object.
\item[\menutwo{Objects}{Change Colour}]  Prompts for a new colour for the
        current object.
\item[\menutwo{Objects}{Change Surface Prop}]  Prompts for an ambient
        coefficient (0--1), a diffuse coefficient (0--1), a
        specular coefficient (0--1), a specular exponent (0--100 or so),
        and an opacity (near 0 is transparent, 1 is fully opaque).
        The currently selected object is assigned these lighting parameters.
\item[\menutwo{Objects}{Invisible}]  Turns the current object invisible.
\item[\menutwo{Objects}{Visible}]  Turns the current object visible.
\item[\menutwo{Objects}{Toggle Visible}]  Toggles the visibility of
        the current object.
\item[\menutwo{Objects}{Next Visible}]  Turns the current object invisible,
               and advances to the next object, making it visible.
\item[\menutwo{Objects}{Prev Visible}]  Turns the current object invisible,
               and advances to the previous object, making it visible.
\item[\menutwo{Objects}{Create Model}]  Creates a model at the current
               position in the hierarchy.  This is useful for grouping
               objects into a single file.
\item[\menutwo{Objects}{Change Model Name}]  Prompts for a name and
               assigns this to the currently selected model in the object
               hierarchy.
\item[\menutwo{Objects}{Cut Object}]  Cuts the currently selected model
               out of the object hierarchy and adds it to the cut buffer.
\item[\menutwo{Objects}{Paste Object}]  Copies the entire cut buffer to
               the current position in the object hierarchy, and clears
\item[\menutwo{Objects}{Flip Object}]  Mirror images the current object
               around the $X = 0$ plane.
\item[\menutwo{Objects}{Scan Object to Volume}]  Causes the intersection
     of the current object with the volume to be displayed in the
     slice window, by assigning the current paint label to any voxel
     which is touching the object.  Works for polygons, lines, and
     markers.
\item[\menutwo{Objects}{Show Vertices}]  Not for general use.
\end{description}

\subsection{3D Render Menu}

\begin{description}
\item[\menutwo{Render}{Change Background}]  Prompts for a colour to
        set the background colour of the 3D window.
\item[\menutwo{Render}{Wireframe or Shaded}]  Toggles the display mode of the
        current object between a wireframe rendering and a solid, shaded 
        rendering.  The default is wireframe.
\item[\menutwo{Render}{Gouraud or Flat}]  Toggles the display mode of the
        current object between a flat or smooth shading.
\item[\menutwo{Render}{Lights}]  Toggles the lights on and off.  If lights
        are off, all objects are coloured uniformly.
\item[\menutwo{Render}{2 Sided}]  Not for general use.
\item[\menutwo{Render}{Backface}]  Not for general use.
\item[\menutwo{Render}{3D DblBuf}]  Toggles the double buffer mode of the
        3D window.  If double buffering is on, the window is smoothly
        updated, but the colour resolution may be poor.
        When double buffering is off, the colour resolution improves, but
        the 3D window flickers when updated.  When taking snapshots,
        turn off double buffering so as to get the best colour resolution.
\item[\menutwo{Render}{Marker Labels}]  Toggles the display of labels
        for the markers in the 3D window.
\item[\menutwo{Render}{Set \# Curve Segments}]  Not for general use.
\end{description}

\subsection{Markers Menu}

\begin{description}
\item[\menutwo{Markers}{Create Marker}]  If the mouse is in the slice window
        over a volume pixel, a marker is created at that location.  Otherwise,
        it is created at the current cursor position.
\item[\menutwo{Markers}{Create Marker}]  If the current object is a marker,
        then changes the marker's position in a manner similar to
        \menutwo{Markers}{Create Marker}.
\item[\menutwo{Markers}{Default Size}]  Prompts the user to type in the default
        marker size in millimeters.
\item[\menutwo{Markers}{Default Label}]  Prompts the user to type in the default
        marker label string.
\item[\menutwo{Markers}{Default Colour}]  Prompts the user to type in the
        default marker colour.
\item[\menutwo{Markers}{Default Structure Id}]  Prompts the user to type in the
        default structure id.
\item[\menutwo{Markers}{Default Patient Id}]  Prompts the user to type in the
        default patient id.
\item[\menutwo{Markers}{Default Type}]  Prompts the user to type in the
        default marker type, of which only cube is supported.
\item[\menutwo{Markers}{Chg Marker Size}]  Prompts the user to type in the
        new size of the current marker, if the current object is a marker.
\item[\menutwo{Markers}{Chg Marker Label}]  Prompts the user to type in the
        new label of the current marker, if the current object is a marker.
\item[\menutwo{Markers}{Chg Marker Type}]  Prompts the user to type in the
        new type of the current marker, if the current object is a marker.
\item[\menutwo{Markers}{Chg Structure Id}]  Prompts the user to type in a
        structure id.  If the current object is a marker, then it is assigned
        this structure id.  If the current object is a model, then all
        markers underneath this object are assigned this structure id.
\item[\menutwo{Markers}{Chg Patient Id}]  Prompts the user to type in a
        patient id.  If the current object is a marker, then it is assigned
        this patient id.  If the current object is a model, then all
        markers underneath this object are assigned this patient id.
\item[\menutwo{Markers}{Move to Marker}]  If the current object is a marker,
        sets the 3D cursor and the volume position to the position of the
        marker.
\item[\menutwo{Markers}{Move Cursor Home}]  Moves the 3D cursor to the
        origin, which should be 0, 0, 0 in Talairach space.  This is useful
        for making slides with the cursor as a reference to AC-PC.
\item[\menuthree{Markers}{Delete Object}{Really Delete}]  Deletes the current object in the
        object hierarchy.
\item[\menutwo{Markers}{Classify Markers}]  Attempts to group markers by
        relative proximity, generating a different colour and structure id
        for each group.
\item[\menutwo{Markers}{Segment Thresh}]  Prompts for and sets the distance
        threshold which determines if two markers are close enough to belong
        to the same group.  Default is 1.5.
\item[\menutwo{Markers}{Pick Modify Marker}]  Allows the user to draw a
        rectangle on the 3D view of the markers and change all the markers
        within the rectangle to the default settings.  This is done by
        pressing and holding the left button, sweeping out a rectangle, and
        when the left button is let go, the defaults are copied to the
        markers.
\item[\menutwo{Markers}{Defaults -$>$ Current}]  Copies the default marker
        values to the current object, if it is a marker.
\item[\menutwo{Markers}{Defaults -$>$ Many}]  Copies the default marker
        values to
        all markers which have the same patient id and structure id as the
        current object, if it is a marker.
\end{description}

\subsection{Polygons Menu}

\begin{description}
\item[\menutwo{Polygons}{Compute Normals}]  Computes normals for the
     current polygon, for use in displaying a smooth lighted surface in
     the 3D window. 
\item[\menutwo{Polygons}{Average Normals}]  A second way to compute
     normals for a polygon, which results in smoother shading.
     Prompts for a number of
     iterations and a ratio value between 0 and 1.  The normals for the
     polygon are computed, then the iterations are performed.  Each
     iteration consists of smoothing each polygon vertex normal with
     its neighbour vertex normals, based on the ratio value.  A value
     of 0 causes no change in each iteration, where a value of 1
     sets each vertex normal to the average of its neighbour vertex
     normals.  Typically, use the values ``10 1'' for this selection.
\item[\menutwo{Polygons}{Set Line Thickness}]  Prompts for a line thickness
     value used in displaying the current polygons in wireframe mode.
     Default value is 1.
\item[\menutwo{Polygons}{Subdivide Polygons}]  Not for general use.
\item[\menutwo{Polygons}{Create Bintree}]  Not for general use.
\item[\menutwo{Polygons}{Reverse Polygons}]  Not for general use.
\item[\menutwo{Polygons}{Reverse Normals}]  Not for general use.
\item[\menutwo{Polygons}{Make Tetrahedral Sphere}]  Not for general use.
\item[\menutwo{Polygons}{Smooth Polygon}]  Not for general use.
\end{description}

\subsection{Surface Curves Menu}

This menu is used to draw curves on surfaces, for the purposes of
delineating sulci and gyri, or for drawing regions on the surface which
can then be coloured.

\begin{description}
\item[\menutwo{Surface Curves}{Start Surf Curve}]  Enters curve drawing
    mode.  Any left mouse click on the surface in the 3D window will
    define a point on the surface.  Subsequent points on the surface are
    connected by a line of shortest distance along the surface.
\item[\menutwo{Surface Curves}{End Surf Curve}]  Exits from curve
    drawing mode.
\item[\menutwo{Surface Curves}{Close Curve}]  When in curve drawing mode
    closes the curve by connecting the end of the curve with the
    start of the curve.
\item[\menutwo{Surface Curves}{Pick Line Point}]  Picks the closest point
    on a line to the point under the mouse, and adds it to the current
    surface curve.
\item[\menutwo{Surface Curves}{Reset Curves}]  Clears the current curve.
\item[\menutwo{Surface Curves}{Permanent Curve}]  Copies the current curve
    into the object hierarchy.
\item[\menutwo{Surface Curves}{Curve Weight}]  Prompts the user to type
    in a curvature weight.  A value of -100 will tend to make surface
    curves follow sulci, and a value of 0 (default) makes it just choose
    shortest path along the surface.
\item[\menutwo{Surface Curves}{Set Crv'tre Limits}]  Prompts the user
    for a minimum and maximum curvature and constrains subsequent curves
    to follow paths along the surface to stay within these surface
    curvature limits.
\end{description}

\subsection{Surface Segmentation Menu}

This menu is used for segmenting the surface into various coloured regions
or to make parts of the surface invisible.  It is still experimental and
is therefore not for general usage.

\end{document}
