\documentstyle[david,psfig]{article}

\title{Display - 3D and Volume Program}
\author{David MacDonald}

\newcommand{\display}{{\bf\tt display}\ }
\newcommand{\menu}[1]{\fbox{\sc\bf #1}}
\newcommand{\menutwo}[2]{\fbox{\sc\bf #1}/\fbox{\sc\bf #2}}
\newcommand{\menuthree}[3]{\fbox{\sc\bf #1}/\fbox{\sc\bf #2}/\fbox{\sc\bf #3}}

%\psdraft

\begin{document}

\maketitle

\newpage

\tableofcontents

\newpage

\section{Introduction}

\display is a program originally designed to display and manipulate
three dimensional objects,
mainly human cortical surfaces and sulcal curves.  It has since evolved to
include MRI and PET volume display and manipulation and a variety of other
features.  The user interface is a non-standard menu oriented system.

\section{Running \display}

\display runs on any of the Silicon Graphics (SGI) workstations:
{\tt portia}, {\tt priam}, {\tt lear}, {\tt duncan}, {\tt grumio},
{\tt phebe}, {\tt quince}, and {\tt ursula}.

The command syntax is:

\begin{verbatim}
    display  [file1]  [file2]  ... [filen]
\end{verbatim}

where each file is one of:

\begin{itemize}

\item[Volume:]  contains an MRI, PET, or other 3D volume, and usually ends
                    in {\tt .mnc},

\item[3D object:]  contains 3D surfaces, lines, or other objects, and usually
                    ends in {\tt .obj},

\item[Landmarks:]  contains a list of 3D points tagged on a volume.  This is
                  an obsolete format and usually ends in {\tt .lmk}, or

\item[Tags:]  contains a list of 3D points tagged on a volume.  This format
             supercedes the landmark format and should be used for all
             lists of points, positions, etc., and usually ends in {\tt .tag}.
\end{itemize}

\subsection{MRI Files at the MNI}

\section{\display Windows}

\postscriptfig{display.ps}{7in}{Clockwise from top left:
3D window, slice window, menu window, text entry window \label{windows}}

Figure \ref{windows} shows the four windows used by \display (three of which
are created when the program is run):

\begin{itemize}
\item[3D window:]  contains the 3D objects, such as surfaces, lines,
                  and markers,
\item[slice window:]  shows 3 orthogonal views of slices through the volume
                     (if one is loaded),
\item[menu window:]  shows the current menu configuration,
\item[text entry:]  this is the shell window or xterm from which \display was
                   run and in which all text must be typed, e.g. filenames,
                   numerical parameters, etc.
\end{itemize}

\section{Menu and Interaction System}

The menu window (bottom right in Figure \ref{windows}) represents the layout
of the left side of the keyboard.  A menu entry is selected by hitting the
corresponding key in any of the 3 display windows or pointing to the entry
with the mouse and clicking the left button.  The space bar or middle mouse
button will pop back one level in the menu.  In the remainder of this document
the following convention will be used to refer to menu selection:
\menutwo{View}{Reset} means select the view menu, then select the reset button
from within the view menu.  Whenever text must be typed, such as in prompts
for filenames, size and width parameters, etc., the mouse must be moved to the
text entry window and the values typed, followed by the return key.
Some menu items cannot be selected by the mouse because the mouse must be
used to point to the object of the action, usually one of the 3 volume slices.
In these cases, the mouse must be positioned on the relevant slice and the
keyboard character corresponding to the menu entry pressed.
When the user is prompted to type in a colour, either the name of a colour,
such as {\tt red}, {\tt yellow}, or {\tt pink}, or a numerical colour,
such as \mbox{\tt 0.3 0.7 0.7} may be typed in.  The following is a brief
description of most useful menu items.  Much of the menu is devoted towards
the author's research and is therefore not applicable to the average 
user, and is not described herein.

\subsection{Menu Window}

The menu window contains the representation of the partial keyboard, and
displays the object hierarchy.  All objects loaded and created in \display,
with the exception of volumes are displayed in the object hierarchy.  The
arrow keys are used to move around in this hierarchy and to select the
current object.

\subsection{3D Window}

In the 3D window, the following mouse operations are available:

\begin{itemize}
\item[Left Button]  Points to a position on a surface or to a marker
                    (tag point).
\item[Middle Button]  Rotates, translates, or magnifies the 3D view, depending
                      on the menu button pressed.
\end{itemize}

\subsection{Slice Window}

In the slice window, the following mouse operations are available:

\begin{itemize}
\item[Left Button]  Sets the position of the 3 slices.
\item[Middle Button]  Moving the mouse while holding down the middle button
                      translates the volume slice.
\item[Right Button]  Paints a region of the volume slice with the current
                     brush size.
\item[Colour Bar]  The limits of the colour bar can be moved by pressing
                   the left mouse button on the bar near the desired limit.
                   Pressing the middle button below the low limit will allow
                   both limits to be moved simultaneously.
\end{itemize}

\section{Menus}

\subsection{Slice Menu}

\begin{description}
\item[\menu{+ Slice}]  Changes the current slice to the next slice.
\item[\menu{- Slice}]  Changes the current slice to the previous slice.
\item[\menutwo{Slice}{Decrease Size}]  Decreases the magnification of the
       volume slice underneath the mouse.
\item[\menutwo{Slice}{Increase Size}]  Increases the magnification of the
       volume slice underneath the mouse.
\item[\menutwo{Slice}{Resample Volume}]  Prompts for 3 size parameter and
       resamples the volume to the given size.  Entering 0 0 0 resets to
       the original volume.
\item[\menutwo{Slice}{Create 3D Slice}]  Creates a rendering of the slice
       under the mouse, in the 3D window.
\end{description}

\subsection{Colour Coding Menu}

\begin{description}
\item[\menutwo{Colour Coding}{Spectral}]  Selects the spectral colour coding
                                          method for the slice volume.
\item[\menutwo{Colour Coding}{Gray Scale}]  Selects the gray scale colour coding
                                          method for the slice volume.
\item[\menutwo{Colour Coding}{Hot Metal}]  Selects the hot metal colour coding
                                          method for the slice volume.
\item[\menutwo{Colour Coding}{Contour}]  Selects the rarely used contour
                                    colour coding method for the slice volume.
\item[\menutwo{Colour Coding}{Range}]  Prompts the user to type in the lower
                                    and upper colour coding limits.
\item[\menutwo{Colour Coding}{Label Ratio}]  Prompts the user to type in the
                      intensity of the coloured labels superimposed on the 
                      volume slices ($0 <= value <= 1$).
\item[\menutwo{Colour Coding}{Under Colour}]  Prompts the user to type in
                                   the colour for values below the low limit.
\item[\menutwo{Colour Coding}{Over Colour}]  Prompts the user to type in
                                   the colour for values above the high limit.
\item[\menutwo{Colour Coding}{Nearest Neighbour}]  Sets the filter type of
                       the slice under the mouse to nearest neighbour (this is
                       the default).
\item[\menutwo{Colour Coding}{Linear Int Filter}]  Sets the filter type of
                       the slice under the mouse to linear interpolation
                       between the two nearest slices.
\item[\menutwo{Colour Coding}{Box Filter}]  Sets the filter type of
                       the slice under the mouse to a box filter.
\item[\menutwo{Colour Coding}{Triangle Filter}]  Sets the filter type of
                       the slice under the mouse to a triangle filter.
\item[\menutwo{Colour Coding}{Gaussian Filter}]  Sets the filter type of
                       the slice under the mouse to a gaussian filter.
\item[\menutwo{Colour Coding}{Filter Width}]  Prompts the user for
                      the filter full width half max.  Only applies to box,
                      triangle, and gaussian filters.
\end{description}

\subsection{Region Painting Menu}

\begin{description}
\item[\menutwo{Region Painting}{Set Paint Label}]  Sets the current label
    used for painting, must be between 0 and 63, where 0 erases.
\item[\menutwo{Region Painting}{Set Paint Label Colour}]  Sets the colour
    of the current label.
\item[\menutwo{Region Painting}{Clear All Labels}]  Clears the labels from
    the entire volume.
\item[\menutwo{Region Painting}{Show Labels}]  Toggles between showing the
    labels and not.
\item[\menutwo{Region Painting}{Save Labels .tag}]  Saves the labels as tag
    points.
\item[\menutwo{Region Painting}{Save Current Label}]  Saves the current label
    as tag points.
\item[\menutwo{Region Painting}{X Radius}]  Prompts for the brush radius used
    in painting on the volume slices (right mouse button).
\item[\menutwo{Region Painting}{Y Radius}]  Prompts for the brush radius used
    in painting on the volume slices (right mouse button).
\item[\menutwo{Region Painting}{Out-Plane Radius}]  Prompts for the brush
    radius in the perpendicular to plane direction.
\end{description}

\subsection{Atlas Menu}

\begin{description}
\item[\menutwo{Atlas}{Atlas State}]  Toggles between displaying scanned images
        of the Talairach Atlas Book superimposed on the volume slices.  The
        first time this is pressed there will be a delay of about 2 minutes
        while the data is read in.
\item[\menutwo{Atlas}{Set Opacity}]  Prompts for the opacity of the atlas.
        Values of 1 will not show the volume slice through the atlas, while
        values near 0 will show a very transparent atlas on top of the volume.
\item[\menutwo{Atlas}{Set Tolerance X}]  Sets the distance from the current
        sagittal slice that atlas pages must be within in order to be
        displayed.
\item[\menutwo{Atlas}{Set Tolerance Y}]  Sets the distance from the current
        coronal slice that atlas pages must be within in order to be
        displayed.
\item[\menutwo{Atlas}{Set Tolerance Z}]  Sets the distance from the current
        transverse slice that atlas pages must be within in order to be
        displayed.
\item[\menutwo{Atlas}{Set Transparent Threshold}]  Not needed.
\item[\menutwo{Atlas}{Flip X}]  Flips the sagittal atlas pages.
\item[\menutwo{Atlas}{Flip Y}]  Flips the coronal atlas pages.
\item[\menutwo{Atlas}{Flip Z}]  Flips the transverse atlas pages.
\end{description}

\subsection{Object Hierarchy}

\begin{description}
\item[\menu{Up Arrow}]  Sets the current object to the one before
                the current object.
\item[\menu{Down Arrow}]  Sets the current object to the one after
                the current object.
\item[\menu{Left Arrow}]  Sets the current object to the one above
                the current object.
\item[\menu{Right Arrow}]  Sets the current object to the one below
                the current object, if the current object is a model.
\end{description}

\subsection{File Menu}

\begin{description}
\item[\menutwo{File}{Load File}]  Prompts for a filename and loads the file.
\item[\menutwo{File}{Save File}]  Prompts for a filename and saves to the file
        the current object, or all objects under the current object, if it is
        a model object.
\end{description}

\subsection{3D Viewing}

The following three menu entries are always available no matter which sub
menu is selected:

\begin{description}
\item[\menu{Magnify}]  Causes the middle mouse button and left/right movement
                       in the 3D window to magnify the view.
\item[\menu{Translate}]  Causes the middle mouse button and movement
                       in the 3D window to translate the objects in the view.
\item[\menu{Rotate}]  Causes the middle mouse button and movement
                       in the 3D window to rotate the objects in a virtual
                       trackball fashion.
\end{description}

\subsection{Markers Menu}

\begin{description}
\item[\menutwo{Markers}{Create Marker}]  If the mouse is in the slice window
        over a volume pixel, a marker is created at that location.  Otherwise,
        it is created at the current cursor position.
\item[\menutwo{Markers}{Create Marker}]  If the current object is a marker,
        then changes the marker's position in a manner similar to
        \menutwo{Markers}{Create Marker}.
\item[\menutwo{Markers}{Default Size}]  Prompts the user to type in the default
        marker size in millimeters.
\item[\menutwo{Markers}{Default Label}]  Prompts the user to type in the default
        marker label string.
\item[\menutwo{Markers}{Default Colour}]  Prompts the user to type in the
        default marker colour.
\item[\menutwo{Markers}{Default Structure Id}]  Prompts the user to type in the
        default structure id.
\item[\menutwo{Markers}{Default Patient Id}]  Prompts the user to type in the
        default patient id.
\item[\menutwo{Markers}{Default Type}]  Prompts the user to type in the
        default marker type, of which only cube is supported.
\item[\menutwo{Markers}{Chg Marker Size}]  Prompts the user to type in the
        new size of the current marker, if the current object is a marker.
\item[\menutwo{Markers}{Chg Marker Label}]  Prompts the user to type in the
        new label of the current marker, if the current object is a marker.
\item[\menutwo{Markers}{Chg Marker Type}]  Prompts the user to type in the
        new type of the current marker, if the current object is a marker.
\item[\menutwo{Markers}{Chg Structure Id}]  Prompts the user to type in a
        structure id.  If the current object is a marker, then it is assigned
        this structure id.  If the current object is a model, then all
        markers underneath this object are assigned this structure id.
\item[\menutwo{Markers}{Chg Patient Id}]  Prompts the user to type in a
        patient id.  If the current object is a marker, then it is assigned
        this patient id.  If the current object is a model, then all
        markers underneath this object are assigned this patient id.
\item[\menutwo{Markers}{Save Mrkrs as .tag}]  Prompts the user to type in a
        filename.  If the current object is a model, all markers underneath
        it are saved to a file in .tag format, which can be later read by 
        \display or the {\tt register} program.  If the current object is
        a marker, then the marker and all markers at this position in the
        object hierarchy are saved to the file.
\item[\menutwo{Markers}{Load Markers}]  Sames as the \menutwo{File}{Load},
        to allow easy access to loading markers.
\item[\menutwo{Markers}{Move to Marker}]  If the current object is a marker,
        sets the 3D cursor and the volume position to the position of the
        marker.
\item[\menuthree{Markers}{Delete Object}{Really Delete}]  Deletes the current object in the
        object hierarchy.
\item[\menutwo{Markers}{Defaults -> Current}]  Copies the default marker
        values to the current object, if it is a marker.
\item[\menutwo{Markers}{Classify Markers}]  Attempts to group markers by
        relative proximity, generating a different colour and structure id
        for each group.
\end{description}

\subsection{View Menu}

\begin{description}
\item[\menutwo{View}{Reset View}]  Resets the 3D window view to the default,
        which corresponds to a top view of the brain surface.
\item[\menutwo{View}{Front View}]  Sets the view to be a view from the front.
\item[\menutwo{View}{Back View}]  Sets the view to be a view from the back.
\item[\menutwo{View}{Left View}]  Sets the view to be a view from the left.
\item[\menutwo{View}{Right View}]  Sets the view to be a view from the right.
\item[\menutwo{View}{Top View}]  Sets the view to be a view from the top.
\item[\menutwo{View}{Bottom View}]  Sets the view to be a view from the bottom.
\item[\menutwo{View}{Fit View}]  Resizes the current 3D image to just fit
                                 within the window.
\end{description}

\subsection{Render Menu}

\begin{description}
\item[\menutwo{Render}{Wireframe or Shaded}]  Toggles the display mode of the
        current object between a wireframe rendering and a solid, shaded 
        rendering.
\item[\menutwo{Render}{Gouraud or Flat}]  Toggles the display mode of the
        current object between a flat or smooth shading.
\item[\menutwo{Render}{Lights}]  Toggles the lights on and off.  If lights
        are off, all objects are coloured uniformly.
\item[\menutwo{Render}{Slice DblBuf}]  Toggles the double buffer mode of the
        slice window.  If double buffering is on, the window is smoothly
        updated, but the colour resolution may be poor.
        When double buffering is off, the colour resolution improves, but
        the slice window flickers when updated.  When taking snapshots,
        turn off double buffering so as to get the best colour resolution.
\item[\menutwo{Render}{3D DblBuf}]  Same as \menutwo{Render}{Slice DblBuf},
        but for the 3D window.
\item[\menutwo{Render}{Marker Labels}]  Toggles between displaying labels
        for the markers in the 3D window.
\end{description}

\subsection{Objects Menu}

\begin{description}
\item[\menuthree{Objects}{Delete Object}{Really Delete}]  Deletes the current
        object.
\item[\menutwo{Objects}{Change Colour}]  Prompts for a new colour for the
        current object.
\item[\menutwo{Objects}{Invisible}]  Turns the current object invisible.
\item[\menutwo{Objects}{Visible}]  Turns the current object visible.
\item[\menutwo{Objects}{Next Visible}]  Turns the current object invisible,
               and advances to the next object, making it visible.
\item[\menutwo{Objects}{Prev Visible}]  Turns the current object invisible,
               and advances to the previous object, making it visible.
\item[\menutwo{Objects}{Create Model}]  Creates a model at the current
               position in the hierarchy.  This is useful for grouping
               markers into a single file.
\item[\menutwo{Objects}{Colour Code Object}]  Colour codes the current object
               with the volume colour coding.  Useful for colour coding an
               MRI surface with a PET volume.
\end{description}

\end{document}
