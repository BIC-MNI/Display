\section{Configuration variables}
\label{secGlobals}

\display{} includes many ``global variables'' that can be set in a
configuration file, from the command line, or using a command within
the program.  These globals serve a wide range of
purposes - they may control the appearance of the user interface, set
parameters for surface extraction, control editing behavior, or even
affect low-level graphics performance. Many of the globals are not
intended to be changed during program execution, as their values may
only be read at startup. Others can be changed dynamically while the
program is running, and may take effect immediately. Some globals are
set implicitly by specific commands in the \display{} user interface.

Most of these globals have not been rigorously tested over their
entire possible range. Input checking is often minimal. It is
therefore possible to alter the behavior of the program in unexpected
or confusing ways. This document will not be comprehensive, but will
attempt to explain some of the most useful variables.

Every global variable has a name and a type, and the program will only
accept values it recognizes as being in the proper type. 

Variable names typically start with an initial capital letter and
consist of several descriptive words separated by underscores. In all
contexts, \display{} requires that the name be given exactly as
specified, with the correct capitalization.

Variable types may be one of the following:
\begin{itemize}
\item String - Any string of ASCII characters.
\item Boolean - A string representing a boolean (true or false) value. Only the first character of the string is examined, with a 't' or `T' indicating true and an 'f' or 'F' indicating false.
\item Integer - A whole decimal number.
\item Real - A decimal number with an optional fractional part.
\item Point - A set of three floating-point number
s specifying a position
along the X,Y, and Z axes.
\item Vector - A set of three floating-point numbers specifying a direction
along the X,Y, and Z axes.
\item Colour - Either a string that gives a colour name (e.g. ``BLUE'') or a series of 3 or 4 numbers specifying RGB or RGBA values in the range 0-1.
\item Surface property - A set of five floating-point parameters that give
the surface properties of an object.
\end{itemize}

Certain types, especially String or Integer, may have other
interpretations depending on the specific variable. For example, some
strings may be interpreted as filenames or formatting strings.

\subsection{Setting globals in a configuration file}
The most reliable way to set global variables is through the configuration file, which is loaded very early in the initialization process. Many of the configuration values are used only during program initialization, so setting them from the command line or the internal command may have no effect!

By default, the configuration file is named {\tt Display.globals}, and
it is most commonly located in the user's home directory. However, the 
program searches for the file in each of the following locations, in order:
\begin{enumerate}
\item /usr/local/lib/
\item The hard-coded installation directory for the package.
\item The directory that contains the binary image for \display.
\item The user's home directory.
\item The current directory ('.').
\end{enumerate}

Note that if a file named {\tt Display.globals} is present in any of these
directories, each such file will be loaded. Therefore it is possible
for later configuration files to add or replace values set in
previously loaded configuration files.

We have recently added a separate configuration file named {\tt
  .mni-displayrc} which is identical in syntax to {\tt
  Display.globals}, but which is intended only for internal use by the
program. It may be overwritten by certain commands in the user interface.

A configuration file consists of a series of lines consisting of a
variable name, followed by one or more spaces, an equals sign ('='),
and finally a variable value. Each line is terminated with a semicolon
(';'). Comments may be included by using a hash ('\#') as the first character
on a line. Comments and empty lines are ignored.

Here is an example {\tt Display.globals} file:
\begin{verbatim}
# Here is a comment.
Default_marker_label = Hippocampus;
#Default_marker_size = 0.02;
Hide_3D_window = false;
Hide_marker_window = false;
Atlas_filename = /data1/users/bert/talairach/Talairach_atlas.list
Max_polygon_scan_distance=0.5;
Object_outline_enabled = True;
Convert_volumes_to_byte=False;
Slice_readout_text_font = 0;
Display_frame_info = True;
\end{verbatim}


\subsection{Setting globals on the command line}

A global can be set from the command line by specifying the {\tt
 -global} command line option. The command line is parsed after the
configuration files are loaded, so values specified on the command
line may override the setting in the configuration files. This will
not be true for all globals, as many of them are only used during the
earliest part of program initialization.

The command line option takes a variable name and value separated from
the option by spaces:
\begin{verbatim}
Display -global Default_paint_label 5 my_t1_image.mnc
\end{verbatim}
This example will set the initial value used for voxel labeling (``painting'') to 5 instead of the default, which is 1.

You can repeat the {\tt -global} option as many times as you like, to
set any number of globals.

\subsection{Setting globals with a menu command}

The \menutwo{Quit}{Global Var} menu command allows you to inspect or
change the value of any global variable. In the default menu structure,
this command is accessed from the ``Quit'' sub-menu, so one can invoke
the command by pressing the '7' key followed by the 'A' key. This will
prompt you to enter a string. If you simply want to check the value of a
variable, you may just enter the variable's name and press Enter. If you
want to change the variable's value, you enter the variable's name
followed by an equals sign ('=') and the desired new value.

Again, be warned that, as with the command line, changing some globals
through this menu command will have no effect. Not all globals will take 
effect after \display{} has been initialized.

\subsection{List of useful globals}

\newcommand{\globalentry}[5]{\subsubsection{{\tt #1}}
\paragraph{Type}{#2}%
\paragraph{Default value}{#3}%
\paragraph{Description}{#4}%
\if\relax\detokenize{#5}\relax%
  %
  \else
  \paragraph{Command}{#5}%
  \fi
}

\globalentry{Atlas\_filename}
{String}{/avgbrain/atlas/talairach/obj/Talairach\_atlas.list}
{Gives the location for the list of Talairach atlas files.}{}

\globalentry{Closest\_front\_plane}
{Real}{1.0e-5}
{Sets the smallest position of the front plane used in 3D rendering.}{}

\globalentry{Convert\_volumes\_to\_byte}
{Boolean}{True}
{If true, volumes are converted to byte as they are loaded, which reduces the possible number of intensities to 256. Set this to false if you need to visualize a file with a higher precision.}{}

\globalentry{Colour\_below}
{Colour}{Black}
{The colour to display for all voxels that lie below the current minimum
 value of the colour coding range.}{}

\globalentry{Colour\_above}
{Colour}{White}
{The colour to display for all voxels that lie above the current maximum
 value of the colour coding range.}{}

\globalentry{Crop\_label\_volume\_threshold}
{Real}{0.9}
{Sets the maximum size ratio for label volume cropping. If cropping is enabled,
 output label volumes will only be cropped if the size of the resulting
 volume is less than this value times the original size.}{}

\globalentry{Cursor\_home}
{Point}{0 0 0}
{Sets the ``home'' location of the cursor in world coordinates, used by 
\menutwo{Markers}{Move Cursor Home}.}{}

\globalentry{Cursor\_rgb\_colour}
{Colour}{Blue}
{Defines the colour of the small box at the centre of the cursor in the 3D window. The colours of the three axes lines are fixed at X=RED, Y=GREEN, Z=BLUE.}{}

\globalentry{Default\_paint\_label}
{Integer}{1}
{The initial value used to label voxels when painting the slice window.}
{{\bf F} Segmenting / {\bf D} Set Paint Lbl}

\globalentry{Default\_x\_brush\_radius}{Real}{3.0}{Default width of the
 primary brush in world units.}{}

\globalentry{Default\_y\_brush\_radius}{Real}{3.0}{Default height of the
primary brush in world units.}{}

\globalentry{Default\_z\_brush\_radius}{Real}{0.0}{Default depth of the
 primary brush in world units.}{}

\globalentry{Draw\_brush\_outline}
{Boolean}{True}
{If true, the outline of the current brush shape will be drawn as voxels are painted in the slice window. If false, only the altered voxel labels themselves will be drawn.}{}

\globalentry{Hide\_3D\_window}
{Boolean}{True}
{If true, the 3D window will be hidden by default. However, if any surface or other geometric objects are loaded, this option will be overridden and the 3D window will be displayed in any case.}{}

\globalentry{Hide\_marker\_window}{Boolean}{True}
{If true, the object list window will be hidden by default. However, if any surface or other geometric objects are loaded, this option will be overridden and the object list window will be displayed in any case.}{}

\globalentry{Hide\_menu\_window}
{Boolean}{False}
{If true, the menu window will be hidden. Command keystrokes will continue to work normally, but the virtual keyboard will not be available.}{}

\globalentry{Histogram\_colour}
{Colour}{WHITE}
{Sets the colour of the histogram curve, as drawn in the slice window if requested.}{}

\globalentry{Initial\_background\_colour}
{Colour}{0.184314, 0.309804, 0.309804} 
{Sets the background colour of each of the windows. Only
the slice window is created late enough for this to have an effect
from the command line, and it has no effect if changed from the menu
command. For some inexplicable reason, the default dark greenish
background is given the symbolic name {\tt DARK\_SLATE\_GREY}. While the
background colour can be changed through a menu command, the command will not
change the colour for the menu or object windows.}{}

\globalentry{Initial\_coding\_range\_absolute}{Boolean}{False}{If true,
the variables Initial\_coding\_range\_low and
Initial\_coding\_range\_high are interpreted as absolute upper and lower
bounds on the initial colour coding range. If false, they are treated as 
fractions of the actual range.}{}

\globalentry{Initial\_coding\_range\_high}{Real}{0.75}{If
  Initial\_coding\_range\_absolute is true, this variable sets the
  initial value of the high range of the colour coding. If
  Initial\_coding\_range\_absolute is false, this value is interpreted
  as a fraction of the actual range.}{}

\globalentry{Initial\_coding\_range\_low}{Real}{0.25}{If
  Initial\_coding\_range\_absolute is true, this variable sets the
  initial value of the low range of the colour coding. If
  Initial\_coding\_range\_absolute is false, this value is interpreted
  as a fraction of the actual range.}{}

\globalentry{Initial\_colour\_coding\_type} {Integer}{1} {Sets the
  colour coding scheme used by the first loaded volume in the slice
  window. The most useful values are 0 for grayscale, 1 for hotmetal,
  13 for spectral, 14 for red, 15 for green, 16 for blue, and 17 for
  contour. Other possibly useful values include 3 for ``cold metal'',
  5 for ``green metal'', 7 for ``lime metal'', 9 for ``red metal'',
  and 11 for ``purple metal''.}{}

\globalentry{Initial\_histogram\_contrast}
{Boolean}{True}
{If true, \display{} will calculate a histogram of each loaded volume
  and use it to set the initial colour coding range. The low range will
  be set attempt to keep Initial\_histogram\_low voxels below the range,
  and Initial\_histogram\_high voxels above the range.}{}

\globalentry{Initial\_histogram\_high}
{Real}{0.99999}
{If this Initial\_histogram\_contrast flag is true, this variable
  determines the percentile used for the initial high colour range.}{}

\globalentry{Initial\_histogram\_low}
{Real}{0.2}
{If this Initial\_histogram\_contrast flag is true, this variable
  determines the percentile used for the initial low colour range.}{}

\globalentry{Initial\_histogram\_low\_clip\_index}
{Integer}{4}
{If this Initial\_histogram\_contrast flag is true, this variable
 determines the number of bins to ignore from the low portion of the 
histogram. In other words, by default we don't consider histogram bins
0-3 to contribute to the percentiles used to calculate the range.}{}

\globalentry{Initial\_perspective\_flag}
{Boolean}{False}
{Sets the initial value of the projection approach used in 3D rendering. A value of false selects parallel projection, whereas true selects perspective projection.}
{{\bf W} 3D view / {\bf D} Proj}

\globalentry{Initial\_render\_mode}
{Boolean}{True}
{If true, 3D objects are rendered in shaded mode by default. If false, 3D objects are rendered in wireframe mode by default.}
{{\bf E} 3D Render / {\bf A} Mode}

\globalentry{Initial\_shading\_type}
{Integer}{1}
{Selects either Gouraud (1) or flat (0) shading as the default used in the 3D window.}
{{\bf E} 3D Render / {\bf S} Shading}

\globalentry{Initial\_slice\_continuity}{Integer}{-1} 
{Selects either nearest-neighbour (-1), trilinear (0), or tricubic (2) for the
 initial slice interpolation method.}{}

\globalentry{Initial\_undo\_feature}
{Boolean}{True}
{Determines the initial state of the ``undo'' option. Set to false if you want undo disabled by default, but wish to leave it under user control. To disable undo completely, set Undo\_enabled to false.}
{{\bf F} Segmenting / {\bf M} Enable Undo}

\globalentry{Menu\_character\_colour} {Colour}{CYAN}
{For active commands, sets the colour of the command name text
  associated with each virtual key in the menu window.}{}

\globalentry{Menu\_character\_inactive\_colour} {Colour}{SLATE\_GREY}
{For inactive commands, sets the colour of the command name text
  associated with each virtual key in the menu window.}{}

\globalentry{Menu\_box\_colour}
{Colour}{WHITE}
{Sets the colour of the box that outlines each virtual key in the menu window.}{}

\globalentry{Menu\_key\_colour}
{Colour}{WHITE}
{Sets the colour of the key name associated with each virtual key in the menu window.}{}

\globalentry{Move\_slice\_speed}
{Real}{0.25}
{Sets the speed at which the slice is changed when the middle button is pressed in the slice window. Use a smaller number to make the slice change more slowly.}{}

\globalentry{Object\_outline\_enabled}
{Boolean}{True}
{If true, any graphics object or surface that is visible in the 3D window will also be displayed by projecting its outline onto the planes of the slice window.}{}

\globalentry{Object\_outline\_width}
{Real}{1.0}
{Sets the line width used when drawing the projection of 3D objects on the planes of the slice window.}{}

\globalentry{Pixels\_per\_double\_size}
{Real}{100.0}
{Sets the number of pixels the cursor must move to double the zoom level when using the original ``shift+middle button'' zoom in the slice window. Smaller numbers yield give faster zooming.}{}

\globalentry{Save\_format} {Integer}{0} {Controls whether graphical
 objects will be saved in ASCII (0) or binary (1) format. In general,
 ASCII format is more likely to be portable among different machine
 architectures.}{}

\globalentry{Secondary\_x\_brush\_radius}{Real}{3.0}{Defines the width
 of the secondary brush in world units.}{}

\globalentry{Secondary\_y\_brush\_radius}{Real}{3.0}{Defines the height
 of the secondary brush in world units.}{}

\globalentry{Secondary\_z\_brush\_radius}{Real}{3.0}{Defines the depth
  of the secondary brush in world units.}{}

\globalentry{Show\_cursor\_contours}
{Boolean}{False}
{If true, the ``cursor contours'' will be displayed in the 3D
  view. These contours are defined by the intersection between the slice
  view planes and the 3D objects.}{}

\globalentry{Show\_markers\_on\_slice}
{Boolean}{True}
{True if Display should draw the outlines of markers in the slice window.}
{}

\globalentry{Show\_slice\_field\_of\_view}
{Boolean}{False}
{If true, displays the current width and height of the field of view of
  each panel in the slice window.}{}

\globalentry{Slice\_change\_fast}
{Integer}{10}
{Sets the amount by which to multiply {\tt Slice\_change\_step} when
  changing slice using the '[' or ']' keys.}{}

\globalentry{Slice\_change\_step}
{Integer}{1}
{Sets the number of slices to move when the '+' or '-' keys are
  pressed.}{}

\globalentry{Slice\_cross\_section\_colour}
{Colour}{GREEN}
{Sets the colour of the slice cross section, the projection of the oblique cross section plane in the three orthogonal planes in the slice window.}{}

\globalentry{Slice\_cursor\_colour1}
{Colour}{RED}
{Sets the color of the ``inner'' part of the slice windows's cursor.}{}

\globalentry{Slice\_cursor\_colour2}
{Colour}{BLUE}
{Sets the color of the ``outer'' part of the slice windows's cursor.}{}

\globalentry{Slice\_divider\_colour}
{Colour}{BLUE}
{Sets the colour of the slice dividers, the lines that define the four quadrants of the slice window.}{}

\globalentry{Slice\_divider\_x\_position}
{Real}{0.5}
{Sets the initial X position of the slice dividers, as a fraction of the overall slice view area.}{}

\globalentry{Slice\_divider\_y\_position}
{Real}{0.5}
{Sets the initial Y position of the slice dividers, as a fraction of the overall slice view area.}{}

\globalentry{Slice\_readout\_text\_font}
{Integer}{0}
{Selects either a fixed-width (0) or proportional-spaced (1) font for the
coordinate and value information displayed in the lower-left corner of the slice window.}{}

\globalentry{Tags\_from\_label}
{Boolean}{False}
{If true, \display{} will attempt to automatically maintain a list of
 tags associated with each label. A marker object will be created
 corresponding an example location that has been marked with each label
 value. A marker will be removed if all of the voxels associated
 with its label value are erased.
 This is still quite experimental and may produce unexpected results.}{}

\globalentry{Undo\_enabled}
{Boolean}{True}
{Determines whether or not the ``undo'' option is {\em ever} enabled when painting the slice window. The undo function requires memory and processing time that may be impractical with very high resolution images. This option silently disabled the undo feature without giving the user the opportunity to turn in back on in the menu.}{}

\globalentry{Use\_cursor\_origin}
{Boolean}{True}
{If true, certain operations, such as rotation of the 3D image, are centered on the current cursor origin. If false, the rotations are centered on the middle of the window.}{}

\globalentry{Use\_zenity\_for\_input}
{Boolean}{True}
{If true, \display{} will attempt to use the Zenity program for user interaction
such as prompts for file names and other user input. If false, 
\display{} will revert to using the terminal window for all user interaction.}{}

\globalentry{Visibility\_on\_input}
{Boolean}{True}
{If true, \display{} will make newly-loaded 3D objects visibile immediately.}{}

\subsection{List of developer globals}
These globals are useful primarily for developers or when debugging problems
with \display. They may be removed or changed with little notice.

\globalentry{Alloc\_checking\_enabled}
{Boolean}{False}
{Enables memory allocation checks. This is primarily intended for use by developers when debugging problems in \display.}{}

\globalentry{Display\_frame\_info}
{Boolean}{False}
{If true, adds an indication of the frame number and elapsed rendering time to the lower left corner of each of the graphics windows. The actual position is determined by the globals {\tt Frame\_info\_x} and {\tt Frame\_info\_y}}{}

\globalentry{Min\_interval\_between\_updates}{Real}{0.02}
{Sets the number of seconds between timer events. Each timer event may
start another redraw operation, so this variable controls the maximum
frame rate of the application.}{}

