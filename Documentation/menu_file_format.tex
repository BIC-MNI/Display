\section{Menu File Format}
\label{secMenuFmt}

In addition to the global variables, \display{} offers complete control over
the layout of the keyboard menus.

On startup, \display{} looks for a file named {\tt Display.menu} in the 
following directories, in order:
\begin{enumerate}
\item The current working directory.
\item The directory where the \display{} application file is located.
\item The value of the HOME environment variable.
\item A system directory, {\tt /usr/local/lib}.
\end{enumerate}
If the {\tt Display.menu} file is found, it will be read and used for the keyboard layout instead of the default internal command menu. However, 
if the environment variable {\tt DISABLE\_MENU} is set, the internal
(hard-coded) menu will always be used.

The {\tt Display.menu} file itself is a text file consisting of a series of {\em menu} definitions, each containing a series of {\em menu entries}. The syntax of a menu is:

{\em menu-name} {\tt '\{'} {\em menu-entry} {\tt '\}'}

and a menu entry has the syntax:

{\em entry-type} {\em key-code} {\em command-or-menu} {\em menu-text} [ {\em help-text} ]

where the {\em entry-type} is either {\tt permanent} or {\tt
  transient}, the {\em key-code} is a quoted string that names a
support key on the keyboard, {\em command-or-menu} is the name of
either a menu or a \display{} menu function, {\em menu-text} is a
short text to display on the keyboard menu itself, and {\em help-text}
is an optional longer text which provides further information about
the menu or command.

The first menu in the file is always the ``root'' or top-level menu.

\subsection{Entry type}
An entry type of {\tt permanent} makes the command is available throughout the menu hierarchy, whereas {\tt transient} commands are only available in their specific menu.

\subsection{Supported key codes}
The key code must be enclosed in quotes.
The supported key codes include all of the standard alphanumeric characters, {\tt 'a'} through {\tt 'z'} and {\tt '0'} through {\tt '9'}. The characters {\tt '+'}, {\tt '-'}, {\tt '='}, {\tt '>'}, {\tt '<'}, {\tt '['}, and {\tt ']'} are also legal command characters.

In addition, the following special key codes are recognized:
\begin{itemize}
\item {\tt f1} -- {\tt f12} - Any of the standard function keys.
\item {\tt left} - The left arrow key.
\item {\tt right} - The right arrow key.
\item {\tt up} - The up arrow key.
\item {\tt down} - The down arrow key.
\item {\tt escape} - The escape key.
\item {\tt delete} - The delete (or 'Del') key.
\item {\tt home} - The home key.
\item {\tt end} - The end key.
\item {\tt pageup} - The page up key.
\item {\tt pagedown} - The page down key.
\item {\tt insert} - The insert key.
\item {\tt ctrl-b} - The Ctrl-B key combination.
\item {\tt ctrl-o} - The Ctrl-O key combination.
\item {\tt ctrl-s} - The Ctrl-S key combination.
\item {\tt ctrl-v} - The Ctrl-V key combination.
\item {\tt ctrl-z} - The Ctrl-Z key combination.
\end{itemize}

\subsection{The command or menu name}
Menu names are specified in the syntax of the {\tt Display.menu} file
itself. Any of these names may be specified as the target of a menu
command. This means that a particular key will push the context of the
menu to the specified submenu.

Otherwise, the name must be a known command function name. There are
many of these commands in \display{}, they are too numerous to list
here, but the details of many of them can be found the in the internal
programmer's documentation , under the 'Menu command function' group.

These strings generally do not contain spaces, and do not need to be quoted.

\subsection{Menu text}
This field provides the short text that will be displayed on the appropriate key
of the simulated keyboard image. The text must appear in quotes.

Many commands do not modify this text in any way, so it can be freely
reconfigured. However, some menu items expect to find formatting codes
(as in { \tt sprintf}) that are used to place values such as current
parameter settings into the text of the menu. In general it is a good idea to 
retain the expected formatting codes ('\%s', '\%d', etc.) in the menu text, even if the surrounding text is modified.
\subsection{Help text}

This optional field specifies a sentence or two which can be displayed
in the menu window when the user places the mouse cursor over the
appropriate keyboard image. This is intended to provide some
additional guidance about the details of each menu or command.
